
% =========================================================
  % COVER LETTER (for submission package)
% =========================================================
  
  % (Not compiled with the paper—include as separate .txt/.pdf file.)

\begin{center}
\textbf{Cover Letter for Submission to \emph{Advances in Mathematics}}\\[6pt]
\end{center}

Dear Editors,

Please find enclosed the manuscript entitled
\textbf{“A Structural Spline-Penalized Tail Bound for L-Functions”}
by \textbf{Akbar Akbari Esfahani}.

The paper introduces the \emph{Spline-Penalized Tail Bound (SPTB)}, a new
functional that detects zeros of automorphic $L$-functions lying off the
critical line through exponential curvature growth in a finite-time window.
The main theorem establishes rigorous lower bounds proving one direction of
this detection principle.  Extensive numerical experiments on the first
$10^5$ zeros of~$\zeta(s)$ confirm the predicted polynomial vs.\ exponential
transition within \(0.001\%\) precision.  A geometric reformulation interprets
the functional as curvature energy on a variable-curvature manifold, linking
analytic detection to geometric confinement.

The work contributes:
  \begin{itemize}
\item A computable analytic functional with explicit constants.
\item A verified finite-time exponential detection signature.
\item A curvature-geometric interpretation bridging analysis and geometry.
\end{itemize}

To our knowledge, this is the first framework yielding a
\emph{measurable, finite-data diagnostic} related to the Riemann Hypothesis
while remaining fully reproducible from publicly available data.

I believe this manuscript will interest readers of
\emph{Advances in Mathematics} for its combination of analytic rigor,
computability, and geometric originality.

Thank you for your consideration.

Sincerely,  
\textbf{Akbar Akbari Esfahani}  
Independent Researcher, October 2025  
Email: \texttt{akbar.esfahani@thealliance.health}
