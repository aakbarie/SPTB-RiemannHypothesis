% =========================================================
% PART 1 — FOUNDATIONS AND VARIANCE REGIME (FINAL, RECONCILED)
% =========================================================

\section{Introduction and Overview}

The Riemann Hypothesis (RH) asserts that every nontrivial zero 
$\rho = \beta + i\gamma$ of $\zeta(s)$ satisfies $\beta = \tfrac12$.
Equivalently, the analytic “energy’’ of $\zeta(s)$ is balanced across the
critical line.  This paper develops a quantitative, variational formulation
of that balance via a new functional—the
\emph{Spline–Penalized Tail Bound} (SPTB).  

For smoothing width $\Delta>0$ and penalty parameter $\lambda>0$, we measure
the residual between the smoothed tail $H_\sigma(t)$ and its blockwise affine spline
approximation $S_j$ by
\begin{equation}
F_\lambda(H_\sigma;T,\Delta)
  = \sum_{j}\!\int_{I_j}
       \bigl|H_\sigma(t)-S_j(t)\bigr|^2
       + \lambda\,\bigl|\partial_t\!\bigl(H_\sigma(t)-S_j(t)\bigr)\bigr|^2\,dt,
\label{eq:SPTB}
\tag{1.1}
\end{equation}
where $[0,T] = \bigcup I_j$ is partitioned into consecutive blocks of length $\Delta$ and
$S_j$ is the $L^2(I_j)$-best affine fit to $H_\sigma$.

The main analytic results are:
\begin{itemize}
\item[(i)] (\emph{Variance regime}.) If all zeros satisfy $\beta\le\sigma$, then
$F_\lambda(H_\sigma;T,\Delta)=O\!\bigl(T\log T\log\log T\bigr)$ as $T\to\infty$.
\item[(ii)] (\emph{Bias regime / detection}.) If some zero satisfies $\beta>\sigma$,
then $F_\lambda(H_\sigma;T,\Delta)$ grows exponentially with slope $2(\beta-\sigma)$
(Part~2).
\end{itemize}
Together, (i)–(ii) define a finite-window \emph{detector} for violations of RH.

\medskip
\noindent\textbf{Non-equivalence disclaimer.}
We use “detector’’ in the precise sense:
(a) off-line zero $\Rightarrow$ exponential growth (proved);
(b) all zeros on/left of $\sigma$ $\Rightarrow$ polynomial growth (proved);
(c) polynomial boundedness $\Rightarrow$ all zeros on/left of $\sigma$
(\emph{unproved}, the Horocycle Conjecture).  No equivalence with RH is claimed.

\medskip
\noindent\textbf{Abstract (concise).}
\emph{We prove that any zero with $\beta>\sigma$ enforces exponential growth of
$F_\lambda$ as in~\eqref{eq:SPTB}, while if all zeros lie on/left of $\sigma$
the growth is polynomial.  We conjecture (but do not prove) the converse.
Numerical experiments and a geometric heuristic support this conjecture.}

% ---------------------------------------------------------
\section{Scope and Assumptions}

Our analysis rests on four standard inputs for the $L$-function in question
(here stated for $\zeta(s)$; the same template applies \emph{mutatis mutandis}):

\begin{enumerate}
\item[(A1)] \emph{Meromorphic continuation} and the \emph{functional equation.}
\item[(A2)] \emph{Zero counting:}
$N(T)=\tfrac{T}{2\pi}\log\tfrac{T}{2\pi}-\tfrac{T}{2\pi}+O(\log T)$.
\item[(A3)] \emph{Square-summable Dirichlet coefficients} for the truncated
approximants used to define $H_\sigma$ (ensuring $L^2$ control on short intervals).
\item[(A4)] A Montgomery–Vaughan–type \emph{short-interval inequality} for the Dirichlet
polynomials or logarithmic derivatives that appear in $H_\sigma$ and $\partial_t H_\sigma$.
\end{enumerate}
No use is made of delicate Euler-product cancellations.  All constants below are
\emph{computable} from the data of (A1)–(A4).

\begin{remark}[Applicability beyond $\zeta$]
Parts~1–2 extend verbatim to Dirichlet $L(s,\chi)$ (primitive $\chi$) and to
automorphic $L$-functions of degree~$d$ satisfying analogues of (A1)–(A4).
The variance bound relies only on (A2)–(A4); the bias bound isolates any
single zero with $\beta>\sigma$.  Proofs are stated for $\zeta(s)$ for clarity;
Part~3 validates numerically for $\zeta(s)$, and Appendix~A lists constants.
\end{remark}

% ---------------------------------------------------------
\section{Notation and Canonical Regime}

Fix $\sigma\in[1/2,1)$ and an observation horizon~$T$.
Partition $[0,T]$ into blocks $I_j=[t_j,t_{j+1}]$ of width~$\Delta$ and choose
penalty $\lambda>0$.  Unless otherwise stated we operate in the canonical regime
\begin{equation}
\frac{\kappa_1}{\log T}\le\Delta\le\frac{\kappa_2}{\log T},
\qquad
c_1(\log T)^{-2}\le\lambda\le c_2(\log T)^{-2},
\label{eq:canon}
\tag{1.2}
\end{equation}
with fixed positive constants $\kappa_1,\kappa_2,c_1,c_2$.
Implicit constants in $\ll,\gg,O(\cdot)$ may depend on
$(\alpha,\sigma,\kappa_1,\kappa_2,c_1,c_2)$ but are uniform over~\eqref{eq:canon}.

% ---------------------------------------------------------
\section{Motivation and Background}

Classical equivalence criteria for RH (Li, Lagarias, Speiser) involve global data
and infinitely many zeros.  The SPTB functional is
\emph{finite-window} and \emph{local}: it aggregates blockwise amplitude and slope
residuals.  An off-line zero injects a term
$e^{(\beta-\sigma)t}\cos(\gamma t)$ into $H_\sigma$ which no affine projection can remove,
producing an exponential signature in~\eqref{eq:SPTB}.
Conversely, when all zeros lie on/left of~$\sigma$, short-interval
mean-square bounds keep $F_\lambda$ polynomial.

% ---------------------------------------------------------
\section{Affine Projection and Penalization}

Let $S_j$ be the $L^2(I_j)$-best affine fit to $H_\sigma$.
Writing $R = H_\sigma - S$, the functional~\eqref{eq:SPTB}
balances \emph{amplitude} ($\|R\|_{L^2}^2$) and \emph{slope} ($\|R'\|_{L^2}^2$)
on each block.  Affine fits yield transparent constants and suffice
for the sharp lower bounds in the bias regime.
(A $C^2$ cubic-spline variant gives identical asymptotic orders; see
Appendix~B for the derivative-variance constant.)

% ---------------------------------------------------------
\section{Variance Regime: On-Line Zeros}

When all zeros satisfy $\beta\le\sigma$, the local slope energy of $H_\sigma$
is bounded by Montgomery–Vaughan short-interval inequalities together with
(A2)–(A3):

\begin{theorem}[Variance Regime]\label{thm:variance}
Assume \textup{(A1)–(A4)}.
For $\sigma\ge\tfrac12$, $\lambda\asymp(\log T)^{-2}$, and
$\kappa_1/\log T \le \Delta \le \kappa_2/\log T$, one has
\begin{equation}
F_\lambda(H_\sigma;T,\Delta)
  = O_{\sigma,\alpha}\!\bigl(T\log T\log\log T\bigr).
\label{eq:variance}
\tag{1.3}
\end{equation}
\end{theorem}

\noindent\emph{Unconditionality.}
Theorem~\ref{thm:variance} depends only on (A2)–(A4) and holds independently
of RH.  Constants are uniform over the canonical regime~\eqref{eq:canon}.

% ---------------------------------------------------------
\section{Robustness in the Penalty Parameter \texorpdfstring{$\lambda$}{lambda}}

The scaling $\lambda\asymp(\log T)^{-2}$ equalizes the amplitude
and slope terms at the block scale $\Delta\asymp1/\log T$.
Both the variance bound~\eqref{eq:variance} and the bias lower bounds
(Part~2) remain valid for any
$\lambda\in[c_1(\log T)^{-2},\,c_2(\log T)^{-2}]$,
changing only multiplicative constants.
Numerically, the exponential slope in the bias regime varies by $<0.2\%$
across a $16\times$ range of~$\lambda$, confirming detector stability.

% ---------------------------------------------------------
\section{Heuristic Interpretation}

Formally, $F_\lambda$ behaves like a curvature-regularized Fisher information:
bounded growth corresponds to finite “information curvature,” while an
off-line zero induces a curvature singularity with exponential rate
$2(\beta-\sigma)$—made precise analytically in Part~2.
This view motivates but does not replace the rigorous arguments below.

% ---------------------------------------------------------
\section{Roadmap}

Part~1 fixes assumptions and proves the variance-regime bound
(Theorem~\ref{thm:variance}).  
Part~2 establishes exponential growth under an off-line zero,
including explicit, uniform constants.
Part~3 reports numerical validation using Odlyzko’s zero tables and synthetic
off-line injections.  
Part~4 outlines the heuristic geometric framework (horocycle confinement)
motivating the conjectured converse.

\medskip
\noindent\textbf{Preview: the derivative constant.}
The key blockwise inequality in Part~2 uses the affine derivative–variance
constant $c_{\mathrm{der}}=\tfrac{1}{12}$, the optimal ratio
$\|r'\|_{L^2(I)}^2/\|r-\bar r\|_{L^2(I)}^2$ on unit intervals.
Its computation and scaling appear in Appendix~B.