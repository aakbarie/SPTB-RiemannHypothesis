
% =========================================================
% PART 1 — FOUNDATIONS AND VARIANCE REGIME (FINAL)
% =========================================================

\section{Introduction and Overview}

The Riemann Hypothesis (RH) asserts that every nontrivial zero 
$\rho=\beta+i\gamma$ of $\zeta(s)$ satisfies $\beta=\tfrac12$.
Equivalently, the analytic ``energy'' of $\zeta(s)$ is balanced across the
critical line.  This paper develops a quantitative, variational formulation
of that balance via a new functional, the
\emph{Spline–Penalized Tail Bound} (SPTB).  For a smoothing width
$\Delta>0$ and penalty parameter $\lambda>0$, we measure the departure of a
blockwise affine spline $S$ from the smoothed tail $H_\sigma(t)$ by
\begin{equation}
F_\lambda(H_\sigma;T,\Delta)
  \;=\; \sum_{j}\!\int_{I_j}
       \bigl|H_\sigma(t)-S_j(t)\bigr|^2
       + \lambda\,\bigl|\partial_t\bigl(H_\sigma(t)-S_j(t)\bigr)\bigr|^2\,dt,
\label{eq:SPTB}
\tag{1.1}
\end{equation}
where $[0,T]=\bigcup I_j$ is partitioned into consecutive blocks of length $\Delta$ and
$S_j$ is the $L^2(I_j)$-best affine fit to $H_\sigma$.

The central analytic results proved in this paper are:
\begin{itemize}
\item[(i)] (\emph{Variance regime}.) If all zeros satisfy $\beta\le\sigma$, then
$F_\lambda(H_\sigma;T,\Delta)=O\!\bigl(T\log T\log\log T\bigr)$ as $T\to\infty$.
\item[(ii)] (\emph{Detection/bias regime}.) If some zero satisfies $\beta>\sigma$,
then $F_\lambda(H_\sigma;T,\Delta)$ grows exponentially in $T$, with asymptotic
slope $2(\beta-\sigma)$ (proved in Part~2).
\end{itemize}
Taken together, (i)–(ii) yield a finite-window \emph{detector} for violations of RH.

\medskip
\noindent\textbf{Non-equivalence disclaimer.}
Throughout we use “detector’’ in the following strict sense:
\emph{(a)} off-line zero $\Rightarrow$ exponential growth (proved);
\emph{(b)} all zeros on/left of $\sigma$ $\Rightarrow$ polynomial growth (proved);
\emph{(c)} polynomial boundedness $\Rightarrow$ all zeros on/left of $\sigma$
(\emph{unproved}, the Horocycle Conjecture).  No equivalence with RH is claimed.

\medskip
\noindent
\textbf{Abstract (concise).}
\emph{We prove that any zero with $\beta>\sigma$ enforces exponential growth of the
SPTB functional \eqref{eq:SPTB}, while if all zeros lie on/left of $\sigma$ the growth
is polynomial.  We conjecture (but do not prove) the converse
(polynomial boundedness $\Rightarrow$ no off-line zeros).  Numerical experiments
and a geometric heuristic support the conjecture.}

% ---------------------------------------------------------
\section{Scope and Assumptions}

Our analysis requires only the following standard hypotheses for the $L$-function under
consideration (here stated for $\zeta$; the same template applies mutatis mutandis):
\begin{enumerate}
\item[(A1)] \emph{Meromorphic continuation} and the \emph{functional equation}.
\item[(A2)] \emph{Zero counting:} $N(T)=\tfrac{T}{2\pi}\log\tfrac{T}{2\pi}-\tfrac{T}{2\pi}
            +O(\log T)$.
\item[(A3)] \emph{Square-summable Dirichlet coefficients} for the truncated approximants used to
            define $H_\sigma$ (ensuring $L^2$ control on short intervals).
\item[(A4)] A Montgomery–Vaughan–type \emph{short-interval inequality} for the Dirichlet
            polynomials/logarithmic derivatives that occur in $H_\sigma$ and $\partial_t H_\sigma$.
\end{enumerate}
We never appeal to delicate Euler-product cancellations.  All constants asserted below are
\emph{computable} from the data of (A1)–(A4).

\begin{remark}[Applicability beyond $\zeta$]
Parts~1–2 extend verbatim to Dirichlet $L(s,\chi)$ (primitive $\chi$) and to standard automorphic
$L$-functions for which analogues of (A1)–(A4) are available.  In particular, the variance bound
relies only on (A2)–(A4), and the bias lower bounds depend on isolating a single zero with
$\beta>\sigma$.  For clarity of exposition we state proofs for $\zeta$; Section~\ref{sec:part3}
provides numerical validation for $\zeta$, and Appendix~A collects the constants used.
\end{remark}

% ---------------------------------------------------------
\section{Notation and Canonical Regime}

We fix $\sigma\in[1/2,1)$, write $T$ for the observation horizon, and partition $[0,T]$
into blocks $I_j=[t_j,t_{j+1}]$ of common width $\Delta$.  The curvature penalty is
$\lambda>0$.  Unless otherwise stated we work in the canonical regime
\begin{equation}
\frac{\kappa_1}{\log T}\;\le\;\Delta\;\le\;\frac{\kappa_2}{\log T},
\qquad
c_1\,(\log T)^{-2}\;\le\;\lambda\;\le\;c_2\,(\log T)^{-2},
\label{eq:canon}
\tag{1.2}
\end{equation}
with fixed positive constants $\kappa_1,\kappa_2,c_1,c_2$.
Implicit constants in $\ll,\gg,O(\cdot)$ may depend on $(\alpha,\sigma,\kappa_1,\kappa_2,c_1,c_2)$,
but are uniform over \eqref{eq:canon}.

% ---------------------------------------------------------
\section{Motivation and Background}

Classical criteria equivalent to RH (Li’s positivity, Lagarias’s inequality, Speiser’s
reformulation) involve global data and—in practice—infinitely many zeros.  The SPTB functional
is \emph{finite-window} and \emph{local}: it aggregates blockwise amplitude and slope residuals.
An off-line zero injects a component of the form $e^{(\beta-\sigma)t}\cos(\gamma t)$ into
$H_\sigma$, which cannot be removed by affine projection and produces an exponential signature
in \eqref{eq:SPTB}.  Conversely, if all zeros lie on/left of $\sigma$, short-interval
mean-square bounds keep $F_\lambda$ polynomial.

% ---------------------------------------------------------
\section{Affine Projection and Penalization}

Let $S_j$ be the $L^2(I_j)$-best affine fit to $H_\sigma$.
Writing $R=H_\sigma-S$, the functional \eqref{eq:SPTB} balances
\emph{amplitude} ($\|R\|_{L^2}^2$) and \emph{slope} ($\|R'\|_{L^2}^2$) per block.
Affine fits make constants transparent, and they suffice for our sharp lower bounds
in the bias regime.  (A $C^2$ cubic spline variant yields the same asymptotic orders; see
Appendix~B for the derivative–variance constant.)

% ---------------------------------------------------------
\section{Variance Regime: On-Line Zeros}

When all zeros satisfy $\beta\le\sigma$, the local slope energy of $H_\sigma$ is controlled by
Montgomery–Vaughan short-interval inequalities together with (A2)–(A3).  We obtain:

\begin{theorem}[Variance Regime]\label{thm:variance}
Assume \textup{(A1)–(A4)}.  For $\sigma\ge\tfrac12$, $\lambda\asymp(\log T)^{-2}$, and
$\kappa_1/\log T \le \Delta \le \kappa_2/\log T$, one has
\begin{equation}
F_\lambda(H_\sigma;T,\Delta)
  \;=\; O_{\sigma,\alpha}\!\bigl(T\log T\log\log T\bigr).
\label{eq:variance}
\tag{1.3}
\end{equation}
\end{theorem}

\noindent
\emph{Unconditionality.}
Theorem~\ref{thm:variance} uses only (A2)–(A4) and is independent of RH.  The constants are
uniform over the canonical regime \eqref{eq:canon}.

% ---------------------------------------------------------
\section{Robustness in the Penalty Parameter \texorpdfstring{$\lambda$}{lambda}}

The normalization $\lambda\asymp(\log T)^{-2}$ balances $\|R\|_{L^2}^2$ and $\|R'\|_{L^2}^2$
at the block scale $\Delta\asymp 1/\log T$.  Both the variance bound \eqref{eq:variance} and the
bias lower bounds proved in Part~2 remain valid for any
$\lambda\in[c_1(\log T)^{-2},\,c_2(\log T)^{-2}]$; only multiplicative constants change.
Numerically, the measured exponential slope in the bias regime varies by $<0.2\%$ across a
$16\times$ range of~$\lambda$, indicating stable detection.

% ---------------------------------------------------------
\section{Heuristic Interpretation}

Formally, $F_\lambda$ behaves like a curvature-regularized Fisher information:
bounded growth corresponds to finite ``information curvature,’’ while an off-line zero induces a
curvature singularity with exponential rate $2(\beta-\sigma)$ (made precise analytically in Part~2).
This viewpoint motivates—but does not replace—the rigorous bounds below.

% ---------------------------------------------------------
\section{Roadmap}

Part~1 fixes assumptions and proves the variance-regime bound (Theorem~\ref{thm:variance}).
Part~2 establishes exponential growth in the presence of an off-line zero, including explicit,
uniform constants.  Part~3 presents numerical validation using Odlyzko’s zero tables and
synthetic off-line injections.  Part~4 offers a heuristic geometric framework (horocycle
confinement) that motivates the conjectured converse.

% ---------------------------------------------------------
\medskip
\noindent
\textbf{Preview: the derivative constant.}
The key blockwise inequality in Part~2 features the affine derivative–variance constant
$c_{\mathrm{der}}=\tfrac{1}{12}$, arising from the optimal ratio
$\|r'\|_{L^2(I)}^2/\|r-\bar r\|_{L^2(I)}^2$ on unit intervals.  Its computation and scaling
are recorded in Appendix~B.
```
