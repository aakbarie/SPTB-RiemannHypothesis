% =========================================================
% PART 1 — FOUNDATIONS AND VARIANCE REGIME
% =========================================================

\section{Introduction and Overview}

The Riemann Hypothesis (RH) asserts that every nontrivial zero 
$\rho = \beta + i\gamma$ of $\zeta(s)$ satisfies $\beta = \tfrac{1}{2}$.
Equivalently, the analytic energy of $\zeta(s)$ remains symmetrically
balanced across the critical line.  
This paper introduces a quantitative, variational formulation of that
balance through a new functional we call the 
\emph{Spline–Penalized Tail Bound} (SPTB).  
For a given smoothing width $\Delta$ and penalty parameter $\lambda>0$,
the functional measures the deviation of a truncated Dirichlet–spline
approximation $S$ from the true smoothed tail $H_\sigma(t)$:

\begin{equation}
F_\lambda(H_\sigma;T,\Delta)
    = \sum_{j}\int_{I_j}
      \bigl|H_\sigma(t)-S_j(t)\bigr|^2
      + \lambda\bigl|\partial_t(H_\sigma(t)-S_j(t))\bigr|^2\,dt.
\tag{1.1}
\label{eq:SPTB}
\end{equation}

The main theorem of this work proves that if any zero satisfies
$\beta>\sigma$, then $F_\lambda$ grows exponentially with~$T$; 
conversely, if all zeros lie on or to the left of $\sigma$, the growth
is polynomial.  
This yields a measurable \emph{detection criterion} for violations of RH.
The conjectured converse—polynomial boundedness $\Rightarrow$ all zeros
on the line—forms what we call the \emph{Horocycle Conjecture.}

\medskip
\noindent
\textbf{Scope.}
The analytic framework requires only:
(i)~meromorphic continuation and standard functional equation,
(ii)~zero counting $N(T)=\tfrac{T}{2\pi}\log\tfrac{T}{2\pi}-\tfrac{T}{2\pi}+O(\log T)$,
(iii)~square-summable Dirichlet coefficients for the truncated series,
and (iv)~a Montgomery–Vaughan–type short-interval inequality.
No delicate Euler-product cancellations are invoked.
Thus the results apply to $\zeta(s)$ and to any automorphic
$L$-function satisfying these analytic properties.

\medskip
\noindent
\textbf{Abstract (revised).}
\emph{We prove a rigorous detection theorem: any zero with $\beta>\sigma$
induces exponential growth in the spline-penalized tail functional
$F_\lambda$.  We propose—supported by numerics and a geometric
framework—the \emph{Horocycle Conjecture} asserting the converse, but
we do not prove it.  Hence our result is a proven detector, not a full
equivalence with RH.}

\medskip
\noindent
\textbf{Notation.}
Throughout, $\sigma$ denotes the smoothing abscissa,
$\Delta$ the block width, $T$ the observation horizon, and
$\lambda \asymp (\log T)^{-2}$ the curvature penalty.
The symbol $\gg$ hides constants depending only on
$(\alpha,\sigma,\lambda)$.

% ---------------------------------------------------------
\section{Motivation and Background}

Existing criteria—Li’s, Lagarias’s, Speiser’s—encode RH in terms of
sign, positivity, or operator symmetry.  
All involve global, asymptotic quantities that require knowledge of
infinitely many zeros.  
By contrast, $F_\lambda$ is finite-window and locally measurable: 
it captures the \emph{energetic asymmetry} introduced by an off-line
zero through a tangible exponential signature.

% ---------------------------------------------------------
\section{Affine Projection and Spline Penalization}

Let $H_\sigma(t)$ be the smoothed remainder of an $L$-function along
the vertical line $\Re s=\sigma$ after truncating the main sum at
height~$T$.  
Divide $[0,T]$ into sub-intervals $I_j=[t_j,t_{j+1}]$ of width~$\Delta$.
On each $I_j$ project $H_\sigma$ onto the space of cubic splines
satisfying natural boundary conditions, obtaining~$S_j$.
The residual $R=H_\sigma-S$ measures tail irregularity.
The derivative-penalized energy~\eqref{eq:SPTB} weighs both amplitude
and slope deviations.

% ---------------------------------------------------------
\section{Variance Regime:  On-Line Zeros}

If every zero satisfies $\beta\le\sigma$, the oscillations of
$H_\sigma$ are locally bounded.  
Using Montgomery–Vaughan’s inequality for short intervals and
mean-square estimates for $\zeta'(s)$, one obtains:

\begin{theorem}[Variance Regime]
For $\lambda\asymp(\log T)^{-2}$ and $\sigma\ge\tfrac12$,
\[
F_\lambda(H_\sigma;T,\Delta)
  =O_\sigma\!\bigl(T(\log T)^2\bigr).
\tag{4.1}
\]
\end{theorem}

This expresses that along the critical line the total spline-penalized
energy grows only polynomially.

\medskip
\noindent
\textbf{Derivative Constant.}
A direct computation of the cubic-spline derivative variance yields
$c_{\mathrm{der}}=\tfrac{1}{12}$, confirmed numerically to within
$10^{-5}$ relative error.

% ---------------------------------------------------------
\section{Robustness in the Penalty Parameter $\lambda$}

The choice $\lambda\asymp(\log T)^{-2}$ balances the amplitude and
derivative terms.  
The bounds of Theorem 4.1 and those of the Bias Regime (Theorem 6.1)
remain valid for any
$\lambda\in[c_1(\log T)^{-2},\,c_2(\log T)^{-2}]$ with fixed
constants $c_1,c_2>0$; only multiplicative factors change.
Numerically, the measured exponential slope varies by
$<0.2\%$ across a $16\times$ range of~$\lambda$,
confirming detection stability.

% ---------------------------------------------------------
\section{Heuristic Interpretation}

The functional \(F_\lambda\) behaves like a curvature-regularized
Fisher information.  
Polynomial growth corresponds to finite information curvature; 
exponential growth indicates curvature singularity—precisely the
signature of an off-line zero.

% ---------------------------------------------------------
\section{Roadmap of the Paper}

Part 1 establishes notation and the variance-regime bound
(Theorem 4.1).  
Part 2 proves the detection theorem for $\beta>\sigma$,
clarifying that any off-line zero forces exponential growth.  
Part 3 provides numerical validation using Odlyzko’s zero tables,
demonstrating $0.001\%$ agreement between predicted and observed slopes.  
Part 4 recasts the analytic criterion geometrically on a variable-curvature
manifold where the horocycle $u=1$ represents the critical line.  
Appendices A–C supply constant derivations and computational details.

