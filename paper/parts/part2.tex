% =========================================================
% PART 2 — BIAS REGIME AND DETECTION THEOREM (RECONCILED FINAL)
% =========================================================

\section{Bias Regime and Detection Theorem}\label{sec:bias}

\subsection{Setup and Decomposition}

Fix $\sigma\in[1/2,1)$ and let $\rho=\beta+i\gamma$ be a zero of $\zeta(s)$ with
$\eta=\beta-\sigma>0$.  We examine its contribution to the smoothed tail
$H_\sigma(t)$ and to the spline–penalized functional $F_\lambda$
from~\eqref{eq:SPTB}:
\begin{equation*}
H_\sigma(t)
  = \sum_{\rho'} \frac{e^{(\beta'-\sigma)t}}{|\rho'|^{\alpha}}\cos(\gamma' t)
  = h_\rho(t)+R(t),
\qquad
h_\rho(t)=|\rho|^{-\alpha}e^{\eta t}\cos(\gamma t),
\end{equation*}
where $R$ collects the contributions of $\rho'\ne\rho$.

\begin{definition}[Single-zero residual]\label{def:single-zero}
For a block $I_j=[t_j,t_{j+1}]$, the \emph{single-zero residual}
associated to $\rho$ is $h_\rho-S_j$, where $S_j$ is the $L^2(I_j)$-best affine fit to $H_\sigma$.
\end{definition}

% ---------------------------------------------------------
\subsection{Step 1: Derivative Lower Bound}\label{step:derivative-lb}

The derivative term in $F_\lambda$ captures slope mismatch that affine fitting cannot remove.
Using the blockwise derivative–variance constant $c_{\mathrm{der}}=\tfrac1{12}$
(Appendix~\ref{app:B}), we obtain
\begin{equation}
\lambda \int_{I_j} |h_\rho'(t)|^2 dt
  \ge \frac{\lambda\,c_{\mathrm{der}}\,\eta^2}{2|\rho|^{2\alpha}}
      \min\!\Bigl\{\Delta,\frac{1}{|\gamma|}\Bigr\}
      e^{2\eta t_j},
\tag{8.1}\label{eq:derivative-lb}
\end{equation}
uniformly in $j$.  Summing over blocks introduces a geometric factor in $e^{2\eta t_j}$.

% ---------------------------------------------------------
\subsection{Step 2: Residual Slope Variance}\label{step:residual-variance}

Let $R=H_\sigma-h_\rho$.  By Cauchy–Schwarz and the variance bound
(Theorem~\ref{thm:variance}),
\begin{equation}
\sum_j\!\int_{I_j}\!|R'(t)|^2dt
   \le C_0\,T\log T\log\log T,
\tag{8.2}\label{eq:residual-variance}
\end{equation}
for an explicit constant $C_0=C_0(\alpha,\sigma)$ depending only on $(\Delta,\lambda)$.

% ---------------------------------------------------------
\subsection{Step 3: Affine Projection Residual Energy}\label{step:affine-energy}

Affine projection removes at most the mean of $h_\rho$ on a block; its oscillatory curvature remains.
For $I_j=[t_j,t_{j+1}]$,
\begin{equation}
\int_{I_j}\!|h_\rho-S_j|^2dt
   \gg e^{2\eta t_j}\!\min\!\Bigl\{\Delta,\tfrac{1}{|\gamma|}\Bigr\}.
\tag{8.3}\label{eq:affine-energy}
\end{equation}
This follows directly from the explicit Gram-matrix computation for
$e^{\eta t}\cos(\gamma t)$ on $[0,\Delta]$ (Lemma 3, Part 1).

\begin{remark}[Spline smoothing]\label{rmk:spline}
Replacing affine fits by natural $C^2$ cubic splines only rescales constants:
for some absolute $c\in(0,1)$,
\[
\int_I |(h_\rho-S_I^{(3)})'|^2 dt
   \ge c\,\int_I |(h_\rho-S_I)'|^2 dt,
\]
uniformly under $\kappa_1/\log T\le\Delta\le\kappa_2/\log T$
and $\lambda\asymp(\log T)^{-2}$.
\end{remark}

% ---------------------------------------------------------
\subsection{Step 4: Geometric Summation}\label{step:geom-sum}

Summing over blocks gives the geometric series
\begin{equation}
\sum_j e^{2\eta t_j}
   = \frac{e^{2\eta T}-1}{e^{2\eta\Delta}-1}
   \asymp \frac{e^{2\eta T}}{4\eta\Delta},
\tag{8.4}\label{eq:geom-sum}
\end{equation}
the source of exponential scaling in $T$.

% ---------------------------------------------------------
\subsection{Step 5: Assembly}\label{step:assembly}

Combining \eqref{eq:derivative-lb}–\eqref{eq:geom-sum} and
subtracting the residual variance~\eqref{eq:residual-variance} yields
\begin{equation}
\lambda\!\sum_j\!\int_{I_j}\!|\partial_t(H_\sigma-S_j)|^2dt
  \ge \frac{\lambda\,c_{\mathrm{der}}\eta^2}{2|\rho|^{2\alpha}}
        \min\!\left\{\!\Delta,\frac{1}{|\gamma|}\!\right\}
        \frac{e^{2\eta T}}{4\eta\Delta}
   - \lambda C_0 T\log T\log\log T.
\tag{8.5}\label{eq:assembly}
\end{equation}
For any fixed $\eta>0$, the first term dominates as $T\to\infty$.

% ---------------------------------------------------------
\subsection{Step 6: Uniform Constants and Threshold Regime}\label{step:uniform}

Tracking constants gives
\begin{equation}
F_\lambda(H_\sigma;T,\Delta)
  \ge \frac{c_{\mathrm{der}}}{8C_0}\,
        \frac{\lambda\eta^2}{|\rho|^{2\alpha}}\,
        \frac{e^{2\eta T}}{\eta\Delta},
\tag{8.6}\label{eq:uniform-lb}
\end{equation}
uniformly for $\kappa_1/\log T \le \Delta \le \kappa_2/\log T$ and
$c_1(\log T)^{-2}\le \lambda \le c_2(\log T)^{-2}$,
with $c_{\mathrm{der}}=\tfrac1{12}$ and $C_0$ from Theorem~\ref{thm:variance}.

\begin{lemma}[Threshold regime]\label{lem:threshold}
Fix the canonical scaling and assume $\eta=\beta-\sigma\ge c/\log T$ for constant $c>0$.
Then for all sufficiently large $T$,
\[
F_\lambda(H_\sigma;T,\Delta)
   \ge A(\alpha,\sigma,c)\,\frac{e^{2\eta T}}{\eta\Delta},
\]
for an explicit $A(\alpha,\sigma,c)>0$.
Hence $F_\lambda\ge\exp(c' T/\log T)$ for some $c'=c'(\alpha,\sigma,c)$:
the exponential regime overtakes any polynomial bound.
\end{lemma}

\begin{proof}[Proof sketch]
Combine~\eqref{eq:derivative-lb} and~\eqref{eq:geom-sum}; note $\eta\Delta\asymp c\kappa/\log^2 T$.
Absorb the polynomial residual using~\eqref{eq:residual-variance}
and choose $T$ large so that $e^{2\eta T}\gg T\log T\log\log T$.
\end{proof}

% ---------------------------------------------------------
\subsection{Main Theorem (Bias/Detection)}\label{subsec:bias-theorem}

\begin{theorem}[Bias regime and detection]\label{thm:bias}
Let $\alpha\ge1$, $\sigma\in(1/2,1)$, and assume the canonical regime
$\kappa_1/\log T \le \Delta \le \kappa_2/\log T$ and
$c_1(\log T)^{-2}\le \lambda \le c_2(\log T)^{-2}$.
If a zero $\rho=\beta+i\gamma$ satisfies $\eta:=\beta-\sigma>0$, then
\[
F_\lambda(H_\sigma;T,\Delta)
   \gg \lambda\,
       \frac{\eta^{2}}{|\rho|^{2\alpha}}\,
       \min\!\Bigl\{\Delta,\tfrac{1}{|\gamma|}\Bigr\}\,
       \frac{e^{2\eta T}}{\eta\Delta},
\]
with an implied constant depending only on
$(\alpha,\sigma,\kappa_1,\kappa_2,c_1,c_2)$.
In particular, any $\beta>\sigma$ forces exponential growth with slope $2(\beta-\sigma)$.
\end{theorem}

% ---------------------------------------------------------
\section{Barrier Equivalence and the Horocycle Conjecture}\label{sec:barrier}

\begin{theorem}[Barrier equivalence: proved and conjectural directions]\label{thm:barrier}
With $F_\lambda$ as above:
\begin{enumerate}
\item[(B$\Rightarrow$A)] If all zeros satisfy $\beta\le\sigma$, then
$F_\lambda=O(T\log T\log\log T)$
(\textup{proved, Theorem~\ref{thm:variance}}).
\item[(A$\Rightarrow$B)] If $\sup_T F_\lambda/(T\log T\log\log T)<\infty$, then
all zeros satisfy $\beta\le\sigma$
(\textup{conjectural, the Horocycle Conjecture}).
\end{enumerate}
\end{theorem}

\begin{conjecture}[Horocycle Conjecture (analytic form)]\label{conj:horocycle}
For every $\sigma\ge\tfrac12$, boundedness of
$F_\lambda(H_\sigma;T,\Delta)/(T\log T\log\log T)$ as $T\to\infty$
implies that all zeros of $\zeta(s)$ satisfy $\beta\le\sigma$.
\end{conjecture}

% ---------------------------------------------------------
\section{Heuristic and Numerical Consistency}\label{sec:numerics}

In the threshold regime $\eta=1/\log T$, $e^{2\eta T}=e^{2T/\log T}$
eventually dominates any polynomial, sharply separating variance and bias behaviour.
The numerical experiments in Part~3 (Odlyzko zeros plus synthetic off-line injections)
show:
(i) polynomial growth under $\beta\le\sigma$, consistent with
Theorem~\ref{thm:variance}; and
(ii) exponential growth with empirical slope matching $2\eta$ to within $10^{-3}$ relative error.

\paragraph{Finite-$T$ caveat.}
For very small $\eta$, the asymptotic slope appears only once $T\gg1/\eta$;
the observed trend is monotone and converges to the theoretical limit.

% ---------------------------------------------------------
\section{Geometric Preview}\label{sec:geom-preview}

The exponential growth proven above is the analytic counterpart of a
radial escape in a variable-curvature model space:
trajectories with $\beta>\sigma$ correspond to $u=e^{\eta t}>1$
and exhibit geodesic-like expansion.
Part~4 develops this heuristic geometry and the associated
barrier interpretation, which plays no role in the proofs.