```latex
% =========================================================
% PART 2 — BIAS REGIME AND DETECTION THEOREM (FINAL)
% =========================================================

\section{Bias Regime and Detection Theorem}\label{sec:bias}

\subsection{Setup and Decomposition}

Fix $\sigma\in[1/2,1)$ and let $\rho=\beta+i\gamma$ be a zero of $\zeta(s)$ with
$\eta=\beta-\sigma>0$.  We study the contribution of $\rho$ to the smoothed tail
$H_\sigma(t)$ and to the penalized functional $F_\lambda$ from \eqref{eq:SPTB}.
Write
\begin{equation*}
H_\sigma(t)
 \;=\; \sum_{\rho'} \frac{e^{(\beta'-\sigma)t}}{|\rho'|^{\alpha}}\cos(\gamma' t)
 \;=\; h_\rho(t)+R(t),
\qquad
h_\rho(t)\;=\;\frac{e^{\eta t}}{|\rho|^{\alpha}}\cos(\gamma t),
\end{equation*}
where $R$ collects the contribution of $\rho'\neq\rho$.

\begin{definition}[Single-zero residual]\label{def:single-zero}
With the notation above, the \emph{single-zero residual} associated to $\rho$ on a block
$I_j=[t_j,t_{j+1}]$ is $h_\rho-S_j$, where $S_j$ is the $L^2(I_j)$-best affine fit to $H_\sigma$.
\end{definition}

% ---------------------------------------------------------
\subsection{Step 1: Derivative Lower Bound}\label{step:derivative-lb}

The derivative term in $F_\lambda$ captures the slope mismatch that affine fitting cannot
remove.  Using the blockwise derivative–variance constant
$c_{\mathrm{der}}=\tfrac1{12}$ (see Appendix~\ref{app:B}), one obtains
\begin{equation}
\lambda \int_{I_j} |h_\rho'(t)|^2 \, dt
  \;\ge\; \frac{\lambda\, c_{\mathrm{der}}\,\eta^2}{2|\rho|^{2\alpha}}
  \min\!\Bigl\{\Delta,\frac{1}{|\gamma|}\Bigr\}
  \,e^{2\eta t_j},
\tag{8.1}\label{eq:derivative-lb}
\end{equation}
uniformly in $j$.  Summing over $j$ will introduce the geometric series in $e^{2\eta t_j}$.

% ---------------------------------------------------------
\subsection{Step 2: Residual Slope Variance}\label{step:residual-variance}

Let $R=H_\sigma-h_\rho$.  By Cauchy–Schwarz together with the variance bound
(Theorem~\ref{thm:variance}), the residual slope energy is at most polynomial:
\begin{equation}
\sum_j\!\int_{I_j}\!|R'(t)|^2\,dt
   \;\le\; C_0\,T\log T\log\log T,
\tag{8.2}\label{eq:residual-variance}
\end{equation}
for an explicit constant $C_0$ depending only on $(\alpha,\sigma)$ and the fixed ranges of
$(\Delta,\lambda)$.

% ---------------------------------------------------------
\subsection{Step 3: Affine Projection Cannot Remove Oscillatory Slope}\label{step:affine-energy}

Affine projection erases at most the mean component of $h_\rho$ on a block; its oscillatory
curvature remains.  For each $I_j=[t_j,t_{j+1}]$,
\begin{equation}
\int_{I_j}\!|h_\rho-S_j|^2\,dt
   \;\gg\; e^{2\eta t_j}\!\int_{I_j}\!\cos^2(\gamma t)\,dt
   \;\asymp\; e^{2\eta t_j}\,\min\!\Bigl\{\Delta,\tfrac{1}{|\gamma|}\Bigr\}.
\tag{8.3}\label{eq:affine-energy}
\end{equation}

\begin{lemma}[Blockwise affine lower bound]\label{lem:affine-lb}
Let $I=[t_0,t_0+\Delta]$ and $h_\rho(t)=|\rho|^{-\alpha} e^{\eta t}\cos(\gamma t)$ with $\eta>0$.
If $S_I$ is the $L^2(I)$-best affine fit to $h_\rho$, then
\[
\int_I |h_\rho - S_I|^2\,dt \;\gg\; e^{2\eta t_0}\,\min\!\Bigl\{\Delta,\tfrac{1}{|\gamma|}\Bigr\},
\]
with an absolute implied constant, uniformly for $\kappa_1/\log T \le \Delta \le \kappa_2/\log T$.
\end{lemma}

\begin{remark}[Cubic smoothing]\label{rmk:cubic}
If $S_I^{(3)}$ is the natural $C^2$ cubic spline minimizing
$\int_I |h_\rho-S|^2+\lambda|(h_\rho-S)'|^2$, then
\(
\int_I |(h_\rho-S_I^{(3)})'|^2\,dt
\ge c\,\int_I |(h_\rho-S_I)'|^2\,dt
\)
for some absolute $c\in(0,1)$, uniformly under
$\kappa_1/\log T \le \Delta \le \kappa_2/\log T$ and $\lambda \asymp (\log T)^{-2}$.
Thus higher-order smoothing cannot remove the slope signal; it changes only constants.
\end{remark}

% ---------------------------------------------------------
\subsection{Step 4: Geometric Summation}\label{step:geom-sum}

Summing block contributions yields the standard geometric sum
\begin{equation}
\sum_j e^{2\eta t_j}
  \;=\; \frac{e^{2\eta T}-1}{e^{2\eta\Delta}-1}
  \;\asymp\; \frac{e^{2\eta T}}{4\eta\Delta},
\tag{8.4}\label{eq:geom-sum}
\end{equation}
which furnishes the exponential scaling in~$T$.

% ---------------------------------------------------------
\subsection{Step 5: Assembly}\label{step:assembly}

Combining \eqref{eq:derivative-lb}–\eqref{eq:geom-sum} and subtracting the residual slope
variance \eqref{eq:residual-variance} gives
\begin{equation}
\lambda\sum_j\!\int_{I_j}\!|\partial_t(H_\sigma-S_j)|^2\,dt
 \;\ge\;
\frac{\lambda\, c_{\mathrm{der}}\,\eta^2}{2|\rho|^{2\alpha}}
 \min\!\left\{\!\Delta,\frac{1}{|\gamma|}\!\right\}
 \frac{e^{2\eta T}}{4\eta\Delta}
 \;-\; \lambda C_0\!\sum_j\!\int_{I_j}\!|R'(t)|^2\,dt.
\tag{8.5}\label{eq:assembly}
\end{equation}

\paragraph{Residual analysis.}
We split into two exhaustive cases.

\smallskip
\noindent
\textbf{Case 1: All residual zeros satisfy $\beta'\le\sigma$.}
Then \eqref{eq:residual-variance} implies the residual term is
$O\bigl(\lambda\,T\log T\log\log T\bigr)$, which is negligible against
the leading factor $e^{2\eta T}/(\eta\Delta)$.

\smallskip
\noindent
\textbf{Case 2: Some residual zero has $\beta'>\sigma$.}
Let $\eta_{\max}=\max_{\rho'\in R}(\beta'-\sigma)$ and choose
$\rho_{\mathrm{worst}}$ attaining it.  The worst residual produces the same
exponential scale as $h_\rho$, and one obtains
\begin{equation}
F_\lambda(H_\sigma;T,\Delta)
 \;\gg\;
 \lambda\,
 \frac{(\max\{\eta,\eta_{\max}\})^2}{|\rho_{\mathrm{worst}}|^{2\alpha}}
 \min\!\Bigl\{\Delta,\tfrac{1}{|\gamma_{\mathrm{worst}}|}\Bigr\}
 \frac{e^{2(\max\{\eta,\eta_{\max}\})T}}{\max\{\eta,\eta_{\max}\}\,\Delta}.
\tag{8.6}\label{eq:worst}
\end{equation}

\begin{lemma}[Exponential dominance prevents cancellation]\label{lem:no-cancel}
Let $u_k(t)=|\rho_k|^{-\alpha} e^{\eta_k t}\cos(\gamma_k t)$ with $\eta_k\ge 0$.
For any coefficients $a_k$ and any $T\ge1$,
\[
\int_0^T \Bigl|\sum_k a_k u_k'(t)\Bigr|^2 dt
\;\ge\; \lambda_{\min}(G_T)\sum_k |a_k|^2 \int_0^T |u_k'(t)|^2 dt,
\]
where $G_T$ is the Gram matrix of the family $\{u_k'\}$ on $[0,T]$.
If $\eta_{\max}>\eta_{(2)}$ by fixed $\delta>0$, then $\lambda_{\min}(G_T)\to1$ as $T\to\infty$.
When several $\eta_k=\eta_{\max}$, the smallest eigenvalue on their span stays $>c(\delta)$.
Thus no cancellation can remove the $e^{2\eta_{\max}T}$ scale.
\end{lemma}

\noindent
Consequently, \emph{any} off-line zero forces exponential growth.

% ---------------------------------------------------------
\subsection{Step 6: Uniform Constants and Threshold Regime}\label{step:uniform}

Tracking constants in \eqref{eq:assembly} and using \eqref{eq:residual-variance} yields
\begin{equation}
F_\lambda(H_\sigma;T,\Delta)
 \;\ge\; \frac{c_{\mathrm{der}}}{8C_0}\;
 \frac{\lambda \eta^2}{|\rho|^{2\alpha}}\;
 \frac{e^{2\eta T}}{\eta\Delta},
\tag{8.7}\label{eq:uniform-lb}
\end{equation}
uniformly for $\kappa_1/\log T \le \Delta \le \kappa_2/\log T$ and
$c_1(\log T)^{-2}\le \lambda \le c_2(\log T)^{-2}$.
Here $c_{\mathrm{der}}=\tfrac1{12}$ (Appendix~\ref{app:B}) and $C_0$ is the constant from
Theorem~\ref{thm:variance}.

\begin{lemma}[Threshold regime]\label{lem:threshold}
Fix the canonical scaling and assume $\eta=\beta-\sigma \ge c/\log T$ with $c>0$ fixed.
Then for all sufficiently large $T$,
\[
F_\lambda(H_\sigma;T,\Delta)\ \ge\ 
A(\alpha,\sigma,c)\cdot \frac{e^{2\eta T}}{\eta \Delta},
\]
for an explicit $A(\alpha,\sigma,c)>0$.  In particular,
$F_\lambda\ge \exp\!\bigl(c' T/\log T\bigr)$ for some $c'=c'(\alpha,\sigma,c)$, so the
exponential regime dominates any polynomial bound.
\end{lemma}

\begin{proof}[Proof sketch]
Combine \eqref{eq:derivative-lb} and \eqref{eq:geom-sum}; use
$\eta\Delta\asymp c\cdot\kappa/\log^2 T$ and absorb the residual by \eqref{eq:residual-variance}.
Choose $T$ large so that $e^{2\eta T}\gg T\log T\log\log T$.
\end{proof}

% ---------------------------------------------------------
\subsection{Main Theorem (Bias/Detection)}\label{subsec:bias-theorem}

\begin{theorem}[Bias regime and detection]\label{thm:bias}
Let $\alpha\ge1$, $\sigma\in(1/2,1)$, and assume the canonical regime
$\kappa_1/\log T \le \Delta \le \kappa_2/\log T$ and
$c_1(\log T)^{-2}\le \lambda \le c_2(\log T)^{-2}$.  If a zero
$\rho=\beta+i\gamma$ satisfies $\eta:=\beta-\sigma>0$, then
\[
F_\lambda(H_\sigma;T,\Delta)
\;\gg\;
\lambda\,
\frac{\eta^{2}}{|\,\rho\,|^{2\alpha}}\,
\min\!\Bigl\{\Delta,\tfrac{1}{|\gamma|}\Bigr\}\,
\frac{e^{2\eta T}}{\eta\Delta},
\]
with an implied constant depending only on $(\alpha,\sigma,\kappa_1,\kappa_2,c_1,c_2)$.
In particular, any $\beta>\sigma$ forces exponential growth with slope $2(\beta-\sigma)$.
\end{theorem}

% ---------------------------------------------------------
\section{Barrier Equivalence and the Horocycle Conjecture}\label{sec:barrier}

\begin{theorem}[Barrier equivalence: proved and conjectural directions]\label{thm:barrier}
With $F_\lambda$ as above:
\begin{enumerate}
\item[(B$\Rightarrow$A)] If all zeros satisfy $\beta\le\sigma$, then
$F_\lambda=O\!\bigl(T\log T\log\log T\bigr)$
\textup{(proved; Theorem~\ref{thm:variance}).}
\item[(A$\Rightarrow$B)] If $\sup_T F_\lambda/(T\log T\log\log T)<\infty$, then all zeros satisfy
$\beta\le\sigma$ \textup{(conjectural: the Horocycle Conjecture).}
\end{enumerate}
\end{theorem}

\begin{conjecture}[Horocycle Conjecture (analytic form)]\label{conj:horocycle}
For every $\sigma\ge\tfrac12$, boundedness of $F_\lambda(H_\sigma;T,\Delta)/(T\log T\log\log T)$
as $T\to\infty$ implies that all zeros of $\zeta(s)$ satisfy $\beta\le\sigma$.
\end{conjecture}

% ---------------------------------------------------------
\section{Heuristic and Numerical Consistency}\label{sec:numerics}

In the threshold regime $\eta=1/\log T$, $e^{2\eta T}=e^{2T/\log T}$ eventually dominates any
polynomial, sharply separating variance and bias behaviours.  The numerical experiments in
Part~3 (Odlyzko zeros plus synthetic off-line injections) exhibit:
(i) polynomial growth under $\beta\le\sigma$ consistent with Theorem~\ref{thm:variance};
(ii) exponential growth with empirical slope matching $2\eta$ to high precision.

\paragraph{Finite-$T$ caveat.}
For very small $\eta$, the asymptotic slope materializes only once $T\gg 1/\eta$; the observed
trend is monotone in $T$ and converges to the predicted limit.

% ---------------------------------------------------------
\section{Geometric Preview}\label{sec:geom-preview}

The exponential growth established here is the analytic shadow of a radial escape in a
variable-curvature model space; trajectories with $\beta>\sigma$ correspond to $u=e^{\eta t}>1$
and exhibit geodesic-like expansion.  Part~4 presents this heuristic geometry and the associated
barrier interpretation.  It is \emph{not} used in the proofs of the theorems above.
```
