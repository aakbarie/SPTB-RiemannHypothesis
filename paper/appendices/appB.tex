% ---------------------------------------------------------
\section{Derivative–Variance (Affine Poincaré) Constant}
\label{app:B}

Let $I=[t_j,t_{j+1}]$ with $|I|=\Delta$ and let $P_{\mathrm{aff}}$ denote the $L^2(I)$
projection onto the affine subspace $\{a+bt\}$. For any $r\in H^1(I)$ satisfying the
orthogonality conditions
\[
P_{\mathrm{aff}} r = 0
\quad\Longleftrightarrow\quad
\int_I r(t)\,dt=0,\qquad \int_I t\,r(t)\,dt=0,
\]
the sharp affine Poincaré (a.k.a.\ Friedrichs) inequality holds:
\begin{equation}
\int_I |r'(t)|^2\,dt \;\ge\; \frac{12}{\Delta^2}\,\int_I |r(t)|^2\,dt.
\label{eq:affine-poincare}
\end{equation}
Equivalently,
\[
\int_I |r(t)|^2\,dt \;\le\; \frac{\Delta^2}{12}\,\int_I |r'(t)|^2\,dt.
\]

\paragraph{Normalization used in Part~2.}
Writing the bound as
\begin{equation}
\int_I |r'(t)|^2\,dt \;\ge\; \frac{1}{12}\,\frac{1}{\Delta^2}\,\int_I |r(t)|^2\,dt,
\qquad c_{\mathrm{der}}=\tfrac{1}{12},
\label{eq:cder}
\end{equation}
matches the scaling adopted in equations~\eqref{eq:derivative-lb} and~\eqref{eq:worst}.

\paragraph{Sharpness and proof sketch.}
The best constant in \eqref{eq:affine-poincare} is the first positive eigenvalue
of the Sturm–Liouville problem for $-d^2/dt^2$ on $I$ with \emph{natural} (Neumann)
boundary conditions and with the function constrained to be $L^2$–orthogonal to the
kernel of the operator restricted to affine functions (i.e., to $\mathrm{span}\{1,t\}$).
After an affine change of variables $t\mapsto x\in[0,1]$ and Gram–Schmidt
orthogonalization against $\{1,x\}$, the minimizer is the unique solution in
$\mathrm{span}\{\cos(\pi x),\sin(\pi x)\}$ that is $L^2$–orthogonal to $\{1,x\}$,
yielding the optimal eigenvalue $12$ on $[0,1]$ and hence $12/\Delta^2$ on $I$.
Equivalently, one may verify optimality by computing
\[
\frac{\int_0^1 (\partial_x \cos(\pi x))^2\,dx}{\int_0^1 (\cos(\pi x)-\overline{\cos})^2\,dx}
= \frac{\pi^2/2}{\pi^2/6} \;=\; 3 \quad\text{for the mean-only constraint,}
\]
and then imposing orthogonality to $x$ increases the Rayleigh quotient by a factor of $4$,
producing the sharp constant $12$; rescaling to length~$\Delta$ gives $12/\Delta^2$.

\paragraph{Discrete grids and numerical check.}
On the experimental block grids used in Part~3, the discrete least-squares projection
onto $\{1,t\}$ and the corresponding finite-difference estimate of $\int_I |r'|^2$
agree with \eqref{eq:affine-poincare} to machine precision (relative error $<10^{-12}$),
confirming both the value $c_{\mathrm{der}}=\tfrac{1}{12}$ and its $\Delta^{-2}$ scaling.

\medskip
\noindent
\emph{Remark.}
Inequality \eqref{eq:affine-poincare} is uniform in the block position~$t_j$
and depends only on the block length~$\Delta$; no assumptions on $r$ beyond
$r\in H^1(I)$ and $P_{\mathrm{aff}} r=0$ are required.
