% ---------------------------------------------------------
\section{Short Appendix: Dirichlet \texorpdfstring{$L(s,\chi)$}{L(s,χ)} Example}
\label{app:dirichlet}

This appendix records the assumptions and standard inputs needed to extend
Theorems~\ref{thm:variance} and~\ref{thm:bias} from $\zeta(s)$ to primitive
Dirichlet $L$–functions $L(s,\chi)$ (mod~$q$). The statements and proofs
carry over \emph{verbatim}, with the same canonical parameter regime
$\Delta\asymp(\log T)^{-1}$ and $\lambda\asymp(\log T)^{-2}$, after replacing
the zero–counting function and the Montgomery–Vaughan short–interval bound by
their Dirichlet analogues and allowing the variance constant $C_0$ to depend
on $q$ only through $\log(qT)$.

\subsection*{Assumptions and analytic normalization}
Let $\chi$ be a primitive Dirichlet character modulo $q\ge 1$, and let
\[
L(s,\chi)=\sum_{n=1}^\infty \frac{\chi(n)}{n^s}, \qquad \Re s>1.
\]
Write the completed $L$–function in the standard form
\[
\Lambda(s,\chi)
=\Bigl(\frac{q}{\pi}\Bigr)^{\frac{s+a}{2}}
\Gamma\!\Bigl(\frac{s+a}{2}\Bigr)\,L(s,\chi),
\qquad
a=\begin{cases}
0,& \chi(-1)=1,\\
1,& \chi(-1)=-1,
\end{cases}
\]
so that $\Lambda(s,\chi)=\varepsilon(\chi)\,\Lambda(1-s,\overline\chi)$ with
$|\varepsilon(\chi)|=1$. The nontrivial zeros $\rho_\chi=\beta_\chi+i\gamma_\chi$
lie in the critical strip and are symmetric with respect to $1/2$ and the real axis.
Along a fixed abscissa $\Re s=\sigma\in[1/2,1)$ we define the smoothed remainder
$H_{\sigma,\chi}(t)$ exactly as in Part~1, with the same truncation and spline
block structure. The Dirichlet–series coefficients $\chi(n)$ are bounded and
square–summable in the truncated ranges used here.

\subsection*{Zero–counting}
Let $N_\chi(T)$ denote the number of zeros $\rho_\chi=\beta_\chi+i\gamma_\chi$
with $0<\gamma_\chi\le T$. For primitive $\chi$ one has the classical formula
\begin{equation}
N_\chi(T)
=\frac{T}{\pi}\log\!\Bigl(\frac{qT}{2\pi}\Bigr)-\frac{T}{\pi}
+O(\log(qT)),
\label{eq:Nchi}
\end{equation}
uniformly in $q\ge 1$ and $T\ge 2$.
This is the direct analogue of the Riemann–von~Mangoldt formula and can be found,
for example, in Davenport~\cite[Ch.~20]{titchmarsh} or Iwaniec–Kowalski.

\subsection*{Short–interval (Montgomery–Vaughan) inequality for Dirichlet polynomials}
For Dirichlet polynomials $D(t)=\sum_{n\le N} a_n n^{-it}$ one has the mean–value
estimate
\[
\int_{x}^{x+H}\!\!\bigl|D(t)\bigr|^2\,dt
\;\le\;
\bigl(H + O(N)\bigr)\sum_{n\le N}|a_n|^2,
\]
uniformly in $x\in\mathbb{R}$ and $H>0$; see Montgomery–Vaughan, \emph{Multiplicative Number Theory~I}, or
the original short–interval inequalities. In our setting, after freezing slowly varying
weights on a block $I_j$ and applying the inequality to $\partial_t H_{\sigma,\chi}$,
one obtains the same per–block control as in Appendix~\ref{app:A}, with the identical
$\min\{\Delta,(\gamma^2 \Delta)^{-1}\}$ structure. Summing over blocks and using
\eqref{eq:Nchi} yields the global variance bound
\begin{equation}
\sum_j \int_{I_j}\!|H'_{\sigma,\chi}(t)|^2\,dt
\;\le\;
C_0(\sigma,\alpha;q)\,T\log(qT)\log\log(qT),
\label{eq:dirichlet-variance}
\end{equation}
where $C_0(\sigma,\alpha;q)=\frac{1}{8\pi^2}\,C^\star_\chi(\sigma,\alpha)$ is explicit and
depends on $q$ only through $\log(qT)$.

\subsection*{Transfer of the variance and bias theorems}
With \eqref{eq:dirichlet-variance} in place, the proofs of Theorems~\ref{thm:variance}
and~\ref{thm:bias} go through without change for $L(s,\chi)$:
\begin{itemize}
\item \textbf{Variance regime.} If all zeros satisfy $\beta_\chi\le \sigma$, then
\[
F_\lambda\bigl(H_{\sigma,\chi};T,\Delta\bigr)
=O_{\sigma,\alpha}\!\bigl(T\log(qT)\log\log(qT)\bigr),
\]
uniformly for $\kappa_1/\log T\le \Delta\le \kappa_2/\log T$ and
$c_1(\log T)^{-2}\le \lambda\le c_2(\log T)^{-2}$.
\item \textbf{Bias/detection regime.} If there is a zero $\rho_\chi$ with
$\eta:=\beta_\chi-\sigma>0$, then the same assembly as in Part~2 yields
\[
F_\lambda\bigl(H_{\sigma,\chi};T,\Delta\bigr)
\;\gg\;
\lambda\,
\frac{\eta^{2}}{|\,\rho_\chi\,|^{2\alpha}}\,
\min\!\Bigl\{\Delta,\tfrac{1}{|\gamma_\chi|}\Bigr\}\,
\frac{e^{2\eta T}}{\eta\Delta},
\]
with implied constants absolute and $C_0$ replaced by $C_0(\sigma,\alpha;q)$.
\end{itemize}

\paragraph{Remark on possible exceptional (Siegel) zeros.}
For real characters $\chi$ there may exist a \emph{Siegel zero}
$\beta_\chi=1-\delta(q)$ with $\delta(q)\ll (\log q)^{-A}$. Such an off–line zero
(when $\sigma<\beta_\chi$) is \emph{exactly} the kind of input detected by the
bias theorem: the exponential rate $2(\beta_\chi-\sigma)$ appears unchanged.
Our arguments do not assume the nonexistence of such zeros; they merely show that
their presence forces exponential growth of $F_\lambda$.

\medskip
\noindent
\textbf{Conclusion.}
All statements of Parts~1–2 extend to $L(s,\chi)$ with the single substitution
$\log T\rightsquigarrow \log(qT)$ in the variance scale and with a possibly different
variance constant $C_0(\sigma,\alpha;q)$; the detection direction and exponential
rate $2(\beta_\chi-\sigma)$ are unchanged.

\medskip
\begin{thebibliography}{9}
\setlength{\itemsep}{2pt}
\bibitem{MV_book}
H.~L.~Montgomery and R.~C.~Vaughan,
\emph{Multiplicative Number Theory I: Classical Theory},
Cambridge Studies in Advanced Mathematics, 2007.

\bibitem{IwaniecKowalski}
H.~Iwaniec and E.~Kowalski,
\emph{Analytic Number Theory},
AMS Colloquium Publications, 2004.

\bibitem{Davenport}
H.~Davenport,
\emph{Multiplicative Number Theory}, 3rd ed., Springer GTM, 2000.
\end{thebibliography}
