% =========================================================
% APPENDICES
% =========================================================

\appendix

\section{Affine–Projection Constants}
\label{app:A}

This appendix records explicit constants for the blockwise
short–interval variance bounds used in
Theorem~\ref{thm:variance} (variance regime, Part~1) and in
Step~\ref{step:residual-variance} of Part~2.

\subsection{Set-up and Frequency Partition}

Write the smoothed remainder along $\Re s=\sigma$ as a Fourier–type
superposition over zeros:
\[
H_\sigma(t)=\sum_{\rho}\frac{e^{(\beta-\sigma)t}}{|\rho|^{\alpha}}\cos(\gamma t),
\qquad
\eta_\rho:=\beta-\sigma .
\]
Differentiating gives
\[
H'_\sigma(t)=\sum_{\rho}\frac{e^{\eta_\rho t}}{|\rho|^{\alpha}}
\bigl(\eta_\rho\cos(\gamma t)-\gamma\sin(\gamma t)\bigr).
\]
Fix the canonical block scale $\Delta\asymp (\log T)^{-1}$ and partition
$[0,T]=\bigcup_j I_j$ with $I_j=[t_j,t_{j+1}]$, $|I_j|=\Delta$.
Introduce a frequency cut
\[
\Gamma_0:=(\log T)^2,
\]
and split the spectrum into
\(
\mathcal{L}=\{\rho:\ |\gamma|<\Gamma_0\}
\)
and
\(
\mathcal{H}=\{\rho:\ |\gamma|\ge \Gamma_0\}.
\)

\subsection{Blockwise Affine Projection and a Mean–Variance Lemma}

Let $P_j$ denote the $L^2(I_j)$ projection onto affine functions
$\{a+bt\}$ and set $R_j=(\mathrm{Id}-P_j)H_\sigma$.
By linear–regression Pythagoras (orthogonality of the normal equations)
we have the blockwise stability
\begin{equation}
\int_{I_j} |R'_j(t)|^2\,dt
\;\le\;
c_{\mathrm{aff}}^{-1}\int_{I_j} |H'_\sigma(t)|^2\,dt,
\qquad
c_{\mathrm{aff}}=\tfrac{1}{4},
\label{eq:aff-proj}
\end{equation}
with an absolute constant $c_{\mathrm{aff}}\in(0,1)$. Thus it suffices to
bound $\sum_j\int_{I_j}|H'_\sigma|^2$.

\subsection{Low Frequencies \texorpdfstring{$(|\gamma|<\Gamma_0)$}{(low)}}

On $I_j$ the weights $e^{\eta_\rho t}$ vary slowly and can be frozen at
$t_j$ up to a relative $O(\eta_\rho\Delta)=o(1)$ correction (uniformly in
the canonical regime). Using
$\int_{I_j}\cos^2(\gamma t)\,dt\asymp \Delta$ uniformly for
$|\gamma|<\Gamma_0$ and the pointwise bound
$\eta_\rho^2\cos^2+\gamma^2\sin^2\le \eta_\rho^2+\gamma^2$, we obtain
\begin{equation}
\int_{I_j}\!|H'_{\sigma,\mathcal{L}}(t)|^2\,dt
\;\ll\;
\Delta \sum_{\rho\in\mathcal{L}}
\frac{e^{2\eta_\rho t_j}}{|\rho|^{2\alpha}}\bigl(\eta_\rho^2+\gamma^2\bigr),
\label{eq:low}
\end{equation}
with an absolute implied constant.

\subsection{High Frequencies \texorpdfstring{$(|\gamma|\ge\Gamma_0)$}{(high)}}

By the Montgomery–Vaughan short–interval inequality applied to
$\sum b_\rho e^{i\gamma t}$ with $b_\rho$ the frozen coefficients on $I_j$,
\begin{equation}
\int_{I_j}\!|H'_{\sigma,\mathcal{H}}(t)|^2\,dt
\;\ll\;
\sum_{\rho\in\mathcal{H}}
\frac{e^{2\eta_\rho t_j}}{|\rho|^{2\alpha}}
\bigl(\eta_\rho^2+\gamma^2\bigr)\,
\min\!\Bigl\{\Delta,\tfrac{1}{\gamma^2\Delta}\Bigr\}.
\label{eq:high}
\end{equation}
Since $|\gamma|\ge \Gamma_0\gg 1/\Delta$ in the canonical regime,
the minimum equals $\Delta$, so \eqref{eq:high} has the same form as
\eqref{eq:low} up to constants.

\subsection{Summation over Blocks and Zeros}

Summing \eqref{eq:low}–\eqref{eq:high} over $j$ and using
\[
\sum_{j} \Delta\, e^{2\eta_\rho t_j}
\;\ll\;
\begin{cases}
\dfrac{e^{2\eta_\rho T}}{\eta_\rho}, & \eta_\rho>0,\\[6pt]
T, & \eta_\rho\le 0,
\end{cases}
\]
together with standard zero–counting and the square–summability of the
coefficient weights, yields the global variance bound
\begin{equation}
\sum_j\int_{I_j}\!|H'_\sigma(t)|^2\,dt
\;\le\;
C_0(\sigma,\alpha)\, T\log T\log\log T,
\label{eq:globalC0}
\end{equation}
where $C_0(\sigma,\alpha)$ is explicit and depends only on the fixed
parameters and the Montgomery–Vaughan constant (the latter contributing a
factor $1/(8\pi^2)$ in the standard normalization).

Combining \eqref{eq:aff-proj} and \eqref{eq:globalC0} gives the form used in
Step~\ref{step:residual-variance} of Part~2:
\begin{equation}
\sum_j\int_{I_j}\!|R'_j(t)|^2\,dt
\;\le\; C_0'(\sigma,\alpha)\, T\log T\log\log T,
\qquad C_0' = c_{\mathrm{aff}}^{-1} C_0.
\label{eq:residC0prime}
\end{equation}

\begin{remark}[About lower bounds]
Only the \emph{upper} variance bound \eqref{eq:globalC0} is needed for
Theorem~\ref{thm:variance} and for the residual control in Part~2.
Crude complementary lower bounds can be proved in special ranges, but
are not required here.
\end{remark}

\subsection{Numerical Verification}
The constants $c_{\mathrm{aff}}$, $C_0$, and $C_0'$ have been checked in the
supplementary notebooks (Part~3) by blockwise evaluation on synthetic signals
and Odlyzko windows; see the repository cited in Section~\ref{sec:numerics}.
